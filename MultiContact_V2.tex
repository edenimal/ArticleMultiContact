%\documentclass[review]{elsarticle}
\documentclass[final,1p]{elsarticle}

\usepackage{lineno}
%\usepackage{hyperref}
\modulolinenumbers[5]

\journal{Journal of \LaTeX\ Templates}

\usepackage{amsmath,amsfonts,mathrsfs,amsbsy,amssymb}
\usepackage{graphicx}
\usepackage{subfig}
\usepackage{color}
\usepackage{endfloat}
\usepackage{epstopdf}
\usepackage{multirow}
\usepackage{pgfplots}
\usepackage{natbib}
\usepackage{tabularx}
\usepackage{booktabs}
\usepackage{adjustbox}
\usepackage{xcolor}
\usepackage{array}
\usepackage{pgf, tikz}
\usepackage{multirow}

\usetikzlibrary{arrows,shapes,positioning}
\newcommand{\midarrow}{\tikz \draw[-triangle 90] (0,0) -- +(.1,0);}
\usepackage[squaren,Gray]{SIunits}
\newcommand{\R}{\mathbb{R}}
\newcommand{\N}{\mathbb{N}}
\newcommand{\REF}{\textcolor{blue}{Mettre une ref}}
\newcommand{\tb}[1]{\textcolor{blue}{#1}}
\newcommand{\bs}[1]{\boldsymbol{#1}}
\newcommand{\mupd}{$\mu_{pad/disc}$}
\newcolumntype{Y}{>{\centering\arraybackslash}X}
\tikzstyle{line} = [draw, color=black]
\tikzstyle{linearrow} = [draw, color=black, -latex']
\tikzstyle{linearrowinv} = [draw, color=black, latex-]




\linespread{1.5}

\graphicspath{{./figures/}}

\begin{document}

\begin{frontmatter}

\title{Determination of the most influential internal contacts on brake squeal based on a genetic algorithm}

\author[LTDS,PSA,a1]{E.~Denimal}
\author[LTDS,IUF,a4]{J.-J.~Sinou}
\author[PSA,a3]{S.~Nacivet}
\author[LTDS,a2]{L.~Nechak}


\address[LTDS]{Laboratoire de Tribologie et Dynamique des Syst\`emes, UMR CNRS 5513,
\'Ecole Centrale de Lyon, 36 avenue Guy de Collongue 69134 \'Ecully Cedex, France}
\address[PSA]{PSA Peugeot Citro\"en, Centre technique de la Garenne Colombes, 18 rue des Fauvelles, 92250 La Garenne Colombes, France}
\address[IUF]{Institut Universitaire de France, 75005 Paris, France}
\address[a1]{enora.denimal@doctorant.ec-lyon.fr}
\address[a2]{lyes.nechak@ec-lyon.fr}
\address[a4]{jean-jacques.sinou@ec-lyon.fr}
\address[a3]{samuel.nacivet@mpsa.com}

%-------------------------------
\begin{abstract}
Many numerical simulations have been dedicated to the brake squeal prediction by considering  the predominant influence of the frictional interface between the pads lining and the disc. However, many others contact interfaces are present in an automotive brake system. Nowadays, the influence of the state of these contacts on the stability behaviour of an automotive brake system and thus, on the squeal propensity is not yet well understood.
This paper attempts to provide a huge investigation on the sensitivity of squeal propensity to the various contact interfaces between pads, piston, bracket and caliper. The main contribution of the present study aims not only to give a better understanding on the role of these numerous contacts in an automotive brake system but also to propose a strategy to be able to reduce the number of simulation to achieve by ranking the effects of all the independent contact states in a pre-design process.

\end{abstract}
%-------------------------------

\begin{keyword}
Automotive brake system, Complex Eigenvalue Analysis, Finite Element Model, Multi-contacts, Friction
\end{keyword}

\end{frontmatter}

\linenumbers

%%%%%%%%%%%%%%%%%%%%%%%%%%%%%%%%%%%%%%%%%%%%%%%%%%%
\section{Introduction}
\label{sec:intro}
%%%%%%%%%%%%%%%%%%%%%%%%%%%%%%%%%%%%%%%%%%%%%%%%%%%

Even if the squeal noise has no effect on the efficiency of braking, the resulting self-excited vibrations and squealing noise from car disc brakes is a source of considerable discomfort and leads to customer dissatisfaction. Therefore,  the growing demands of having quieter brakes and the customer complaints that may result in significant warranty costs motivate the need to study brake squeal early in the design process. Thereby a better understanding and prediction of the propensity of squeal noise in car brake systems  is an important challenge that is faced by the automotive engineers. The classical methodology to study friction-induced squeal in a complete automotive disc brake assembly is based on the well-known Complex Eigenvalues Analysis (CEA) that is performed around non-linear equilibrium points of the mechanical system under study. From a complex eigenvalue extraction the dynamic instabilities (i.e. the appearance of initial growing oscillatory behaviour) that create unwanted squeal noise are predicted. In general, one explanation given for the brake squeal phenomenon is the coupling of two modes during brake operation: the friction between the pads and the disc induces the dynamic instability in the automotive brake system.

Even if the determination of  the assembled state of the brake system that secure the brake pads in the assembly is sometimes estimated in the first analysis step, the ability to use a representative model that takes into account the entire possible contact states  between different part of the brake assembly independently is only rarely implemented. This can be explained by supposing that the frictional contact between the pads lining and the disc is  one of the prime  factor in regard to squeal noise. Some studies offer sometimes further analysis on the influence of the friction coefficient by considering various effects such as a non-uniform contact pressure between the brake linings and disc, the dependence of the friction coefficient versus velocity or temperature or the pads wear (even though many studies use a constant value of the friction coefficient). Therefore a rigorous methodology to study friction-induced squeal in a complete automotive disc brake assembly by considering the impact of the rubbing or sliding contact state on the secondary internal interfaces has not been proposed so far. The consequences of these contact states on the squeal propensity is quite poorly known. So the improved understanding of considering internal contacts is of prior importance for the industrial and scientific community working on the automotive brake squeal.

The paper is organized as follows. Firstly, the motivation of the present paper and the automotive brake system under study with the specificity of taking into account several frictional or sliding contact interfaces are presented. Then the classical CEA is briefly described. Secondly, Section \ref{subsec:preliminary} gives preliminary results and conclusions on the influence of the friction coefficient between pads and the disc as well as other potential frictional interfaces. Then, a deeper understanding of the role of internal contact interfaces is proposed in Section \ref{sec:betterunderstanding} by undertaking analysis of evolutions of the unstable modes as well as the influence of the contact conditions per frequency range. Then, the  restriction on the concept of the most unstable mode per frequency range is introduced in order to rank the effects of all the independent contact states. Finally, a strategy based on a Genetic Algorithm is adopted to define with a reduced computational cost  the most unfavourable case (depending on the different internal contact states) to be considered in a design study in regard to  the squeal propensity. Motivation of such numerical strategy for an industrial point of view, as well as the advantages and limitations of the proposed methodology will be discussed.



%%%%%%%%%%%%%%%%%%%%%%%%%%%%%%%%%%%%%%%%%%%%%%%%%%%
\section{Automotive brake system and stability analysis}
\label{subsec:FEMandCEA}
%%%%%%%%%%%%%%%%%%%%%%%%%%%%%%%%%%%%%%%%%%%%%%%%%%%
In this section, the Finite Element Model (FEM) of the automotive brake system under study will be first presented. Then brief reminders about the stability analysis are given.

%%%%%%%%%%%%%%%%%%%%%%%%%%%%%%%%%%%%%%%%%%%%%%%%%%%
\subsection{Finite element model of the brake system and motivation of the proposed study }
%%%%%%%%%%%%%%%%%%%%%%%%%%%%%%%%%%%%%%%%%%%%%%%%%%%

The brake system under study is shown in Fig.~\ref{fig:FEM}. The hydraulically actuated disc brake consists of several components such as the disc, floating caliper, outer and inner pads, hub knuckle, piston and wheel. The braking action can be summarized as follows: during the braking process, hydraulic pressure is applied to the piston, and the inner pad is pushed by the piston until it contacts the disc. Then, the reaction forces push the caliper and the outer pad against the opposite side of the disc. The caliper that holds the two pads can move in parallel to the rotation axis of the disc when pressurizing and braking. This action create friction between the pads and disc. Sometimes squeal occurs when the brakes are applied. Even if it should not negatively affect brake stopping performance, brake squeal may generated vibration of the brake components, especially the pads and disc. 

Traditionally the friction coefficient at the pad/disc interface is considered as one of the most influential parameter on the automotive brake system stability. Thereby, the majority of numerical simulations for the prediction of squeal propensity are performed by considering only the two friction surfaces between the pads and disc for different values of this friction coefficient and neglecting other friction surfaces at other components of the automotive brake system.

However, many contact interfaces may be present in an automotive brake system. During braking, different components can interact with each other through different contact interfaces. It is particularly true for the pads that can be in contact with three components in addition to the disc, namely the piston, the caliper and the bracket. The status of each contact is directly related to the position of the pads in the automotive brake system, which are supposed to be free of movement. It can be noted that a new static equilibrium of the brake system may appear if a state of contact is modified (in particular if the contact is considered as sliding or not). This leads to a new potential state of the brake system with respect to its stability. 

The implementation and the validation of the full FE model is not the main concern of this study. For more details, please see \cite{AAA}. We will only explain the inclusion of various contact states in the following. Considering the FEM under study, the three contact interactions taken into account (i.e. piston/pad, bracket/pad and caliper/pad) lead us to consider nine new possible contact interfaces, as indicated in Figure \ref{fig:PadsContact} (see the letters A, B, C, D E, F G, H and I) and summarized in Table~\ref{tab:PadsContact}. 

Considering these three contact interactions between the piston and the pad, the bracket and the pad and the caliper and the pad), the non-linear dynamic problem of the brake system can be written as:
\begin{equation}
\bf{M}\ddot{\textbf{X}} + \bf{C}\dot{\textbf{X}} + \bf{KX} +  \bf{F}_{piston/pad}(\bf{X}) + \bf{F}_{bracket/pad}(\bf{X})+ \bf{F}_{caliper/pad}(\bf{X}) + \textbf{F}_{nl}(\textbf{X}) = \bf{F}_{ext} 
\label{eq:eq1}
\end{equation}

where $\bf{X}$ is the displacement, and the dot represents the derivative with respect to time. $\bf{M}$, $\textbf{C}$ and $\textbf{K}$ are mass, damping and structural stiffness matrices, respectively. $\bf{F}_{ext}$ represents the external efforts, here it corresponds to the pressure applied on the piston and the caliper. $\bf{F}_{piston/pad}(\textbf{X})$, $\bf{F}_{bracket/pad}(\textbf{X})$ and $\bf{F}_{caliper/pad}(\bf{X})$ define the equivalent stiffness contributions for the interfaces of the two sub-systems under consideration. In the case of contact for a given interface, normal forces and friction forces are generated. The detachment (i.e. switching from a contact state to a non-contact state at the selected interfaces) is allowed for each interface. $\bf{F_{nl}}$ are the non-linear efforts and contains contributions from both the non-linear contact forces and frictional forces at the pad/disc contact interfaces. Each contact interface  is characterized by a classical Coulomb's law with a constant friction coefficient. 

So the motivation and the goal of the present paper is to study the impact of considering all of these contact surfaces on the automotive brake system stability. 


%%%%%%%%%%%%%%%%%%%%%%%%%%
\begin{figure}[h!]
\centering
\includegraphics[height=.25\textheight]{FEM.eps}
\caption{Finite element model developed with Abaqus - Assembled view (left) and exploded view (right)}
\label{fig:FEM}
\end{figure}
%%%%%%%%%%%%%%%%%%%%%%%%%%
%%%%%%%%%%%%%%%%%%%%%%%%%%
\begin{figure}[h!]
\centering
\includegraphics[height=.15\textheight]{contact_plaquette.eps}
\caption{Contact Interfaces of internal (left) and external (right) pads}
\label{fig:PadsContact}
\end{figure}
%%%%%%%%%%%%%%%%%%%%%%%%%%

%%%%%%%%%%%%%%%%%%%%%%%%%%
\begin{table}[h!]
\centering
\caption{Contact parameters}
\begin{tabularx}{\hsize}{cXll}
\toprule
\textbf{Interface letter} & \textbf{Type of contact} &  \multicolumn{2}{c}{\textbf{Parameter}}\\
\midrule
A & Inner Top Radial Contact  && RHI \\
B & Inner Bottom Radial Contact && RBI \\
C & Inner Bracket-Pad Contact && RCBI \\
D & Pad-Piston && PP \\
E & Outer Bottom Radial Contact && RBE \\
F & Outer Top Radial Contact && RHE \\
G & Outer Bracket-Pad Contact && RCBE \\
H & Bottom Caliper-Pad Contact && DB \\
I & Top Caliper-Pad Contact && DH \\
\bottomrule
\end{tabularx}
\label{tab:PadsContact}
\end{table}
%%%%%%%%%%%%%%%%%%%%%%%%%%

%%%%%%%%%%%%%%%%%%%%%%%%%%%%%%%%%%%%%%%%%%%%%%%%%%%
\subsection{Stability analysis}
%%%%%%%%%%%%%%%%%%%%%%%%%%%%%%%%%%%%%%%%%%%%%%%%%%%

To predict instabilities of mechanical systems subjected to friction-induced vibrations, two complementary methodologies can be used. The first one is the Complex Eigenvalues Analysis (CEA), and the second one is the dynamic transient and stationary analysis, each with its own advantages and disadvantages. As explained in \cite{Sinou_MRC}, the calculation of the transient and/or stationary self-exited vibrations is the most relevant process to detect self-excited vibrations and brake squeal. However, this method is often computationally extremely expensive and is not easily affordable in the case of large industrial systems. In this context, strategies based on CEA are often used to predict the stability of the system in a given frequency range \cite{Kinkaid}.  Even if this method can lead to an under- or over-estimation of the unstable modes present in the  time simulations, this method is the most popular deterministic approach to predict squeal instability. This is of common use nowadays, especially in car industry. One of the major drawbacks of CEA is that it is valid just in the neighbourhood of the static equilibrium point (i.e. the local aspect of the performed prediction) and it cannot deal with the contact area changes of the frictional interfaces with time during vibration.

The prediction of friction-induced instabilities by the CEA is based on the analysis of the eigenvalues of the system linearised around its static equilibrium. The presence of the friction forces may introduce non-symmetric terms in the system's matrices, which can lead to instabilities. The full CEA procedure is detailed bellow.

By defining the global stiffness matrix $\bf{K_{nl}}$ due to not only the structural components of the brake system but also the three contact interfaces as follow: 
\begin{equation}
\bf{K_{nl}X}=\bf{KX} + \bf{F}_{piston/pad}(\bf{X}) + \bf{F}_{bracket/pad}(\bf{X})+ \bf{F}_{caliper/pad}(\bf{X}) 
\label{eq:jj1}
\end{equation}
the determination of the non-linear static equilibrium $\bf{U}_S$ is given by:
\begin{equation}
\bf{K_{nl,\bf{U}_S}}\bf{U_S} +\bf{ F}_{nl}(\bf{U}_S) = \bf{F}_{ext}
\label{eq:eq2}
\end{equation}
where $\bf{K_{nl,\bf{U}_S}}$ corresponds to the linearized stiffness matrix at the vicinity of the non-linear equilibrium point. This contribution includes the possibility of contact or loss of contact for the three internal contact interfaces between the pad and the piston, the bracket and the pad, and the caliper and the pad.

Then, the system is studied around the equilibrium point $\bf{U_S}$ by assuming a small perturbation $\overline{\textbf{X}}$  (i.e. $\bf{X} = \bf{U_S} + \overline{\bf{X}}$). Considering the linearisation of the the non-linear force $\bf{F}_{nl}$ around $\bf{U}_S$ 
(i.e. $\bf{F_{nl}(\bf{X}) = \textbf{F}_{nl}(\bf{U}_S) + \bf{J}_{nl} \overline{\bf{X}} }$), the previous relation of Equation \ref{eq:eq1} can be rewritten in the following form:
\begin{equation}
\textbf{M} \ddot{\overline{\bf{X}}} + \bf{C} \dot{\overline{\bf{X}}} + (\bf{K_{nl,\bf{U}_S}} + \bf{J}_{nl})\overline{\textbf{X}} = \bf{0}
\label{aq:eq4}
\end{equation}

The eigenvalue problem to be solved is then given by  eigenvalue problem:
\begin{equation}
(\lambda^2  \bf{M} + \lambda \textbf{C} + (\bf{K_{nl,\bf{U}_S}}+ \bf{J}_{nl}))\boldsymbol{\Psi} = \bf{0}
\end{equation}

Due to friction, the matrix $\bf{J_{nl}}$ is non-symmetric. So running CEA produces complex eigenvalues $\lambda_j= a_j + i\omega_j$ and complex eigenvectors $\boldsymbol{\Psi}_j$. $\omega_j$ defines the pulsation of the mode $\boldsymbol{\Psi}_j$ and $a_j$ the associated real part. The system is considered as unstable if at least one eigenvalue has a positive real part. If all the real parts are negative  the equilibrium point remains stable.



%%%%%%%%%%%%%%%%%%%%%%%%%%%%%%%%%%%%%%%%%%%%%%%%%%%
\section{Preliminary results and conclusions on the influence of the friction coefficient at internal contact interfaces}
\label{subsec:InfluenceInternalContacts}
%%%%%%%%%%%%%%%%%%%%%%%%%%%%%%%%%%%%%%%%%%%%%%%%%%%

A classical methodology to study the squeal propensity of an automotive brake system consists in running several CEA for different values of the friction coefficient. Therefore, in order to have a first trend on the influence of the friction coefficient (denoted $\mu_{intern}$) at the nine internal contact interfaces, the following strategy is first proposed: all the considered internal contacts presented in Table \ref{tab:PadsContact} are characterized by the same friction coefficient $\mu_{intern}$. Two specific cases are undertaken:
\begin{itemize}
	\item Case 1: \tb{a variation of the friction coefficient $\mu_{intern}$ is considered} at the nine internal contact interfaces (from 0 to 1 by keeping the same value for each internal interface) for three different values of $\mu_{pad-disc}$: $\mu_{pad-disc} = 0.3$, $\mu_{pad-disc} = 0.5$ or $\mu_{pad-disc} = 0.7$;
		\item Case 2: four specific configurations versus $\mu_{intern}$ are investigated with a variation of the friction coefficient $\mu_{pad-disc}$ from 0 to 1.  The four different values of the friction coefficient $\mu_{intern}$ at the nine internal contact interfaces are  $\mu_{intern}=0$, $\mu_{intern}=0.15$, $\mu_{intern}=0.3$ and $\mu_{intern}=0.5$. It is reminded that the nine internal interfaces have the same value of $\mu_{intern}$ for each configuration.
\end{itemize}

The evolution of the eigenvalues in the complex plan are displayed in Figure \ref{fig:ETUDE1} for Case~1. Similarly, Table \ref{tab:InsCarac} summarizes the number of instabilities, the value of the critical point $\mu_{pad-disc}$ and the associated frequency of the unstable mode for the four configurations tested in Case 2.

All these preliminary results of Cases 1 and 2 demonstrate the influence of the friction coefficient $\mu_{intern}$. Showing Figure \ref{fig:ETUDE1}, it is also observed that even if the variation of $\mu_{intern}$ does not drastically affect the frequency value of the unstable modes, a significant effect is visible on the variation of the real parts. For instance, if $\mu_{pad-disc} = 0.5$, at $3900$ Hz, the real part varies from about $30$ to about $290$. Of course as previously discussed, Figure \ref{fig:ETUDE1} illustrates the fact that the friction coefficient $\mu_{intern}$ impacts the number of unstable modes. Hence, when $\mu_{pad-disc} = 0.5$, at about $5300$ Hz, the mode is unstable only when $\mu_{intern}$ is superior to $0.6$. 
Comparing cases where  $\mu_{intern} \neq 0$, the main difference is observed on the real part of the instabilities and on the number of instabilities. When $\mu_{intern}$ increases, the number of unstable modes increases. As summarized in Table \ref{tab:InsCarac}, when $\mu_{intern}=0.15$ only four instabilities are present, whereas six instabilities (seven respectively) exist when $\mu_{intern}=0.3$ ($\mu_{intern}=0.5$ respectively). Moreover, a variation of $\mu_{intern}$ implies a significant variation of the real parts of unstable modes. Indeed, considering the instability around $[4700-4800]$ Hz, the maximum of the real part may decrease or increase when $\mu_{intern}$ increases according to the value of $\mu_{pad-disc}$ (see and compare Figures \ref{fig:ETUDE1} for example). 

It is also observed in Table \ref{tab:InsCarac} that the value of the critical point and the frequency of each unstable mode are very dependent on both the values of $\mu_{intern}$ and $\mu_{pad-disc}$. So the effect of $\mu_{intern}$ can not be considered as negligible. It is not intuitive and it can be complex to analyse. However, by analysing all the results, a first trend emerges. The most notable differences are generally attributable to the fact that the friction coefficient $\mu_{intern}$ is different from or equal to zero comparing the configuration $\mu_{intern} =0$ with the other configurations (i.e. $\mu_{intern} \neq 0$). In other words, two distinguishable groups can be made. The first one corresponds to the eigenvalues obtained under sliding contact conditions (i.e. $\mu_{intern} =0$), and the second to those obtained under frictional contact conditions (i.e. $\mu_{intern} \neq 0$). 

Now a first intermediate conclusion can be made: this preliminary study demonstrates the influential role of internal contacts of a brake system on its stability. It is obvious that in a process of characterizing the squeal propensity of a brake system, they cannot be neglected. In view of the numerical cost that so many additional parameters may imply, this conclusion can be a tedious one and a strategy to take these contacts into account at a reasonable cost is necessary. As previously discussed, the first strong trend lies on the fact that the system can be studied with two different states for each internal contact, namely a sliding one or a frictional one. In this preliminary study we recall that we considered a simplified case for which all internal contact interfaces behave in the same way. In a more general context where each internal contact can have a specific configuration (sliding with $\mu_{intern} =0$ or friction with $\mu_{intern} \neq 0$), a number of $2^9=512$ different configurations (i.e. nine internal contacts and two states) for one fixed $\mu_{pad-disc}$ has to be running which is already a large number of CEA to perform for an industrial finite element model. 

At this stage of the study it is essential to remember that the configuration $\mu_{intern} \neq 0$ can be performed by using different friction values $\mu_{intern}$. This choice can then impact the stability results. Also in order to carry out a more detailed study on the independent effect of each internal contact interface, we propose to  quickly explain the choice that has been made on the value of $\mu_{intern}$ for the rest of the study. In order to argue our choice the evolution of the magnitude of the total force due to contact pressure at the different internal interfaces are displayed on Figure~\ref{fig:ETUDE1_Force}. It is observed that the magnitude of the total pressure in each contact changes and reaches a steady state above a certain threshold. An increasing of $\mu_{pad-disc}$ raises this threshold. For some contacts, the variation between the initial state at $\mu_{intern} = 0$ and the steady state is weak, as for PP or MU int/ext. But for others, the variation can be large and becomes larger when $\mu_{pad-disc}$ increases, as for RCBE which has an initial total magnitude of pressure forces equals to $1550$ N when $\mu_{intern} =0$ and about $500$ N when $\mu_{intern}=1$, if $\mu_{pad-disc}=0.7$. In light of these results, to choose the value of the non-zero friction coefficient $\mu_{intern}$, two different choices are possible. The first one is a friction coefficient $\mu_{intern}$ on which the system has reached its steady state. This choice has the advantage to describe the system in a steady state. But, the steady state's threshold varies with $\mu_{pad-disc}$  and so it could be difficult to determine a value for which the system is always at its steady state, not to mention that this threshold value is initially unknown and may depend of the contact. The second choice is to take a value of the friction coefficient $\mu_{intern}$ for which the system is always in a transitional state. This choice has the advantage to describe the system when the system is changing and it is easier to determine a value for which the system is always in a transient state. It is this second choice that we retained for the rest of the study. The value of the friction coefficient $\mu_{intern}$ is taken equal to $0.15$. Even if this choice appears to be close to the threshold value if $\mu_{pad-disc}=0.3$, the design of automotive brake systems are generally performed for values of the friction coefficient $\mu_{pad-disc}$ superior to $0.3$. Therefore this choice is consistent in relation to our motivation and the purpose of our study.


%%%%%%%%%%%%%%%%%%%%%%%%%%
%\begin{figure}[h!]
%\centering
%\includegraphics[height=.6\textheight]{Effets_coeff_balayages.eps}
%\caption{Evolution of eigenvalues of the brake system in the complex plan assuming a variation of the friction coefficient at the pad-disc interface for different values of the friction coefficient of internal contacts (blue: $\mu_{intern} = 0$, red: $\mu_{intern} = 0.15$, yellow: $\mu_{intern} = 0.3$, purple: $\mu_{intern} = 0.5$) }
%\label{fig:EffetCoeff}
%\end{figure}
%%%%%%%%%%%%%%%%%%%%%%%%%%



%%%%%%%%%%%%%%%%%%%%%%%%%%%%%%%%%%%%%%%%%%%%%%%%%%%%%%%%%%%%%%%%%%%%%%%%%%%%%%%%%%%%%%%%%%%%%%%%%%
% Table 2
% JJS : A modifier comme 1ere ligne
% je ne comprends pas pourquoi parfois les intervalles [ ; ] sont 1 et 1 en bornes min et max....
%%%%%%%%%%%%%%%%%%%%%%%%%%%%%%%%%%%%%%%%%%%%%%%%%%%%%%%%%%%%%%%%%%%%%%%%%%%%%%%%%%%%%%%%%%%%%%%%%%
%%%%%%%%%%%%%%%%%%%%%%%%%%
\begin{table}[h!]
\centering
\caption{Instabilities for different internal friction coefficients}
\begin{tabularx}{\hsize}{YYYYY}
\toprule
			   &  \textbf{Instability number } & \textbf{Frequency (Hz)}    & \textbf{Maximum Real Part}     & $\left[\boldsymbol{\mu}_{\textbf{c,inf}};\boldsymbol{\mu}_{\textbf{c,sup}}\right] $ 	  \\
\midrule
\multirow{13}{*}{$\boldsymbol{\mu}_{\textbf{intern}}=\textbf{0}$} 	
											& 1		& 540.3		&133.3		& [0.1;1]			\\
											& 2		&676.7		& 62.92		& [0.9;1]			\\
											& 3		&1574		&33.14		& [0.2;1]			\\
											& 4		& 2172		& 216.9		& [0.75;1]			\\
											& 5		& 3431		& 842.8		& [0.45;1]			\\
											& 6		& 3461		& 350.1		& [0.7;1]			\\
											& 7		& 3662		& 181.3		& [0.35;0.55]		\\
											& 8		& 4168		& 624.7		& [0.7;1]			\\
											& 9		& 4239		& 236.5		& [0.25;0.45]		\\
											& 10		& 4242		& 168.8		& [0.8;0.95]		\\
											& 11		& 4598		& 72.81		& [0.9;1]			\\
											& 12		& 4742		& 141.6		& [1;1]				\\
											& 13		& 5355		& 374.9		& [0.7;1]			\\
\midrule
\multirow{5}{*}{$\boldsymbol{\mu}_{\textbf{intern}}=\textbf{0.15}$} 
											& 1		& 3119		& 92.08		& [1;1]			\\
											& 2		& 3446		& 34.52		& [1;1]			\\
											& 3		& 3633		& 16.06		& [0.6;0.6]		\\
											& 4		& 3810		& 192.8		& [0.35;1]		\\
											& 5		& 5389		& 658.2		& [0.4;1]		\\
\midrule
\multirow{6}{*}{$\boldsymbol{\mu}_{\textbf{intern}}=\textbf{0.3}$} 
											& 1		& 803.4		& 40.52		& [0.65;1]	 	\\
											& 2		& 3106		& 100.9		& [0.65;1]	 	\\
											& 3		& 3443		& 48.27		& [1;1]	 		\\
											& 4		& 3669		& 30.38		& [1;1]			 \\
											& 5		& 3816		& 178		& [0.35;1]		 \\
											& 6		& 5365		& 627.6		& [0.65;1]	 	\\
\midrule
\multirow{8}{*}{$\boldsymbol{\mu}_{\textbf{intern}}=\textbf{0.5}$} 
											& 1		& 802.9		& 72.43		& [0.65;1]	 	\\
											& 2		& 3002		& 32.95		& [0.8;1]	 	\\
											& 3		& 3093		& 82.75		& [0.6;1]	 	\\
											& 4		& 3681		& 227.5		& [0.75;1]	 	\\
											& 5		& 3824		& 145.3		& [0.4;1]	 	\\
											& 6		& 4331		& 104.3		& [0.65;1]	 	\\
											& 7		& 5241		& 133.9		& [0.85;0.95]	 \\
											& 8		& 5324		& 302.7		& [1;1]	 		\\
\bottomrule
\end{tabularx}
\label{tab:InsCarac}
\end{table}
%%%%%%%%%%%%%%%%%%%%%%%%%%


%%%%%%%%%%%%%%%%%%%%%%%%%%
\begin{figure}[tb]
%\begin{changemargin}{-4cm}{-4cm}
	\centering
	\begin{tabular}{@{}cc@{}}
	\subfloat[a][]{
	\includegraphics[height=.3\textheight]{mu1.eps}
	\label{fig:ETUDE1_mu1}}&
	\subfloat[b][]{
	\includegraphics[height=.3\textheight]{mu2.eps}
	\label{fig:ETUDE1_mu2}}\\
	\multicolumn{2}{c}{\subfloat[c][]{
	\includegraphics[height=.3\textheight]{mu3.eps}
	\label{fig:ETUDE1_mu3}}}\\
	\end{tabular}
%	\end{changemargin}
	\caption{Evolution of complex eigenvalues in the complex plan assuming a variation of the friction coefficient $\mu_{intern}$ at the interfaces for three different values of $\mu_{pad-disc}$: (a) $\mu_{pad-disc} = 0.3$, (b) $\mu_{pad-disc} = 0.5$ (c) $\mu_{pad-disc} = 0.7$}
	\label{fig:ETUDE1}
\end{figure}
%%%%%%%%%%%%%%%%%%%%%%%%%%

%%%%%%%%%%%%%%%%%%%%%%%%%%
\begin{figure}[tb]
%\begin{changemargin}{-4cm}{-4cm}
	\centering
	\begin{tabular}{@{}cc@{}}
	\subfloat[a][]{
	\includegraphics[height=.29\textheight]{Efforts_pression_mu_1.eps}
	\label{fig:ETUDE1_mu1_F}}&
	\subfloat[b][]{
	\includegraphics[height=.29\textheight]{Efforts_pression_mu_2.eps}
	\label{fig:ETUDE1_mu2_F}}\\
	\subfloat[c][]{
	\includegraphics[height=.29\textheight]{Efforts_pression_mu_3.eps}
	\label{fig:ETUDE1_mu3_F}} &
	\multicolumn{1}{l}{\subfloat{
	\includegraphics[scale=0.7]{Legend_pression.eps}
	\label{fig:ETUDE1_Pression_legend}}}\\
	\end{tabular}
%	\end{changemargin}
	\caption{Evolution of the magnitude of the total force due to contact pressure at the different interfaces versus the friction coefficient $\mu_{intern}$ at the interfaces for three different values of $\mu_{pad-disc}$: (a) $\mu_{pad-disc} = 0.3$, (b) $\mu_{pad-disc} = 0.5$ (c) $\mu_{pad-disc} = 0.7$}
	\label{fig:ETUDE1_Force}
\end{figure}
%%%%%%%%%%%%%%%%%%%%%%%%%%



%%%%%%%%%%%%%%%%%%%%%%%%%%%%%%%%%%%%%%%%%%%%%%%%%%%
\section{On a better understanding of the role of internal contact interfaces}
\label{sec:}
%%%%%%%%%%%%%%%%%%%%%%%%%%%%%%%%%%%%%%%%%%%%%%%%%%%

%%%%%%%%%%%%%%%%%%%%%%%%%%%%%%%%%%%%%%%%%%%%%%%%%%%
\subsection{Motivation}
\label{sec:}
%%%%%%%%%%%%%%%%%%%%%%%%%%%%%%%%%%%%%%%%%%%%%%%%%%%


Based on the above results and to stay in affordable computational times, the choice can be made to limit the study to the consideration of two contact states for each of the nine interfaces: a sliding state with a zero friction coefficient  $\mu_{intern}$, and a frictional state with a friction coefficient $\mu_{intern}$ equals to $0.15$. If all combinations of internal contacts are considered for a fixed $\mu_{pad-disc}$, significantly differences on the squeal propensity results can nevertheless be observed. This is illustrated in Figure \ref{fig:PlanCplx512}: in this case the friction coefficient $\mu_{pad-disc}$ at the pad/disc interface is constant and fixed at $0.5$. The CEA associated to the $512$ possible configurations are performed and the obtained eigenvalues are superimposed and displayed in the complex plan. The influence of the contact conditions is clear and important. Some groups of eigenvalues can be made: for example between $1080$ and $1120$ Hz or between $1850$ and $2000$ Hz. But for frequencies higher than $3300$ Hz, a strong dispersion is observed and it is not trivial to follow the different unstable modes versus the different configurations. A first conclusion is the high sensitivity of the unstable eigenvalues to contact states. So it is necessary to investigate more precisely those results by analysing precisely the impact of each of the nine internal contact interfaces on the generation or not of unstable modes.

In order to conduct such an analysis, this section is organized as follow: in a first time, the unstable modes are studied and compared. In a second time, the contact conditions associated to an unstable frequency are analysed. The main objective is to identify recurrent unstable modes, or to identify contacts that are systematically involved in some instabilities.

%%%%%%%%%%%%%%%%%%%%%%%%%%
\begin{figure}[h!]
\hspace{-3cm}
\includegraphics[height=.6\textheight]{Plan_cplx_512cfg.eps}
\caption{Complex eigenvalues for all considered configurations at $\mu_{pad/disc}=0.5$}
\label{fig:PlanCplx512}
\end{figure}
%%%%%%%%%%%%%%%%%%%%%%%%%%


%%%%%%%%%%%%%%%%%%%%%%%%%%%%%%%%%%%%%%%%%%%%%%%%%%%
\subsection{Analysis of the unstable modes versus internal contacts}
\label{sec:}
%%%%%%%%%%%%%%%%%%%%%%%%%%%%%%%%%%%%%%%%%%%%%%%%%%%

The first step consists in the comparison of the unstable modes. For this purpose, a group of $25$ representative configurations is retained. Their characteristics are given in Table~\ref{tab:Carac25CFG} and the unstable frequencies are given in Table~\ref{tab:FreqIns25CFG}. For information, the first configuration corresponds to the case where all internal contacts are sliding and the second where all are frictional.  The eleventh and twelfth configurations correspond to the two most unstable configurations (i.e. with the maximum real part). For this study, the friction coefficient at the pad/disc interface  $\mu_{pad-disc}$ is chosen equal to $0.9$ in order to get as many instabilities as possible.

The different unstable modes observed on the twenty-five configurations tested are compared by applying a Modal Assurance Criterion (MAC). The MAC corresponds to a measure of the degree of linearity between two modes. A high value of the MAC indicates that the mode shapes are proportional, and so modes can be paired and it is possible to consider that they correspond to the same instability. A low value, indicates that mode shapes are orthogonal, and so that they do not correspond to the same instability. The MAC value between two mode shapes $\bs{\Phi}$ and $\bs{\Phi}_2$ is given by: 
\begin{equation}
\text{MAC}(\bs{\Phi}_1,\bs{\Phi}_2)=\left( \frac{\vert \bs{\Phi}_1^{*} \bs{\Phi}_2 \vert}{\Vert \bs{\Phi}_1 \Vert \Vert \bs{\Phi}_2 \Vert}\right) ^2
\label{eq:MAC}
\end{equation}
where $^{*}$ designate the conjugate transpose of a complex vector. 

For the sake of simplicity, the unstable mode shapes of the twenty-five configurations tested are concatenated and an AutoMAC is performed. The AutoMAC matrix obtained is displayed Figure \ref{fig:MAC25CFG}. The different configurations are separated by a red dot line. The block at the $i^{\text{th}}$ row and at the $j^{\text{th}}$ column corresponds to the MAC matrix between the unstable modes of configurations $i$ and $j$. The blocks on the diagonal are the AutoMAC of each configuration. 

Except for a few exceptions, there is no correspondence between the different unstable modes of the different configurations. Some configurations seem to have similar unstable modes (for example configurations $2$ and $19$ or $25$ and $8$), but in most cases there is no correlation. The Frequency MAC (FMAC), which is a MAC with a frequency scale, is displayed Figure \ref{fig:FMAC25CFG}. It appears that mode shapes with a high correlation have close natural frequencies. Indeed, MAC values around 1 are at proximity of the diagonal. But it is also possible to find two modes with similar natural frequencies and totally different mode shapes. This study shows that the unstable modes are different for each configuration and are sensitive to the contact state at each of the nine internal interfaces. It means that it is actually not possible to track an unstable mode depending on the different configurations, and so to make a group of unstable frequencies associated to the same unstable mode. 

A further study was carried out by comparing all modes (i.e. stable and unstable) of the different configurations and the results were similar. It points out two important conclusions: one unstable mode for a specific configuration does not correspond to a stable mode of another configuration; and the whole modal basis is influenced by the contact states. 


To investigate more precisely the problem, the MAC is also computed by considering just one component of the system. The different MAC matrix of each component are displayed Figure \ref{fig:MACElem}. For the sake of visibility, results are presented only for the first twelve configurations, but they are similar for the others. If only the disc is considered (see Figure \ref{fig:MACElem} (a)), numerous correlations are observed. In other words the vibration modes of the disc are common to the different configurations. However, when a pairing can be done through the mode shape of the disc, it appears that there is no correlation between the other components, particularly in the case of the caliper and the bracket. This two components are highly sensitive to the contact states at the nine internal interfaces, and a pairing just by tracking the mode shapes of the disc could lead to a misunderstanding of the system behaviour. To illustrate this observation, some mode shapes are represented and compared: for two different cases, the mode shapes of two modes which have a high MAC disc value are given in Figure ~\ref{fig:CompCFG5_14} for the first case, and in Figure ~\ref{fig:CompCFG3_4} for the second case. Their characteristics are given in Table~\ref{tab:ModesComparesCarac}. For the first case, the unstable mode is located around $3480$ Hz for configurations $5$ and $14$. The global MAC is equal to $0.5748$, but the MAC associated to the disc is equal to $0.9555$. Indeed, when the deformed shapes of the disc are compared, they are visually the same for both configurations. But if the focus is done on the caliper or on the bracket, the deformed shapes are different. It is confirmed by the two associated MAC values equal to $0.053$ and $0.088$ respectively. For configuration $5$ (see Figure~\ref{fig:CompCFG5_14}(a) and (c)), there is almost no deformation on the bracket and maximum of amplitude are concentrated on the disc. But for configuration $14$ (see Fig.~\ref{fig:CompCFG5_14}(b) and (d)), the bracket has an important deformation which is of the same order as the disc.

In the same way, by comparing the instability at $4250$ Hz of configurations $3$ and $4$, the disc has the same mode shape which is consistent with the high MAC value for the disc that is equal to $0.9444$. But when looking more precisely the other components, their mode shapes are different. This is particularly true in the case of the bracket component, for which the MAC value is equal to $0.0423$.  For both configurations, the maximum of deformation is localized on the handle (see Figure \ref{fig:CompCFG3_4}). But for configuration $4$, the deformation at the contact with the screw on the right is also important (see Figure~\ref{fig:CompCFG3_4}(d)), whereas it is not the case for configuration $3$ (see Figure ~\ref{fig:CompCFG3_4}(c)). 

In conclusion, these results lead to an important difficulty to deal with the different contacts conditions for the nine internal interfaces. It would have been interesting to identify one or several unstable modes common to the different configurations. Thus, the unstable eigenvalues could have been grouped together by tracking an unstable mode. However, the modes are sensitive to the contact states for each of the nine internal interfaces. So it is not possible to follow an unstable mode with respect to the different configurations. It is then necessary to adopt an other strategy to analyze the problem under study and to have an increased understanding of the role of each internal contact interfaces on the squeal propensity.




%%%%%%%%%%%%%%%%%%%%%%%%%%%%%%%%%%%%%%%%%%%%%%%%%%%%%%%%%%%%%%%%%%%%%%%%%%%%%%%%%%%%%%%%%%%%%%%%%%
% Table 3 
% JJS : A modifier comme 1ere ligne
% mettre le nombre d'instabilités au début (et pas à la fin....)
% au début contac tname -> configuration
\begin{table}[h!]
\centering
\caption{Characteristics of the $25$ configurations under study}
\small
\hspace*{-3.5cm}
\begin{tabularx}{20cm}{cc*{9}{X}}
\toprule
\multirow{2}{*}{\textbf{Configuration}} &\multirow{2}{2cm}{\textbf{Number of instabilities}} & \multicolumn{9}{c}{\textbf{Contact Name}} \\
 & & 	\textbf{RHI}		&\textbf{RHE}		&\textbf{RCBE}		& \textbf{RCBI}	& 	\textbf{RBE}	& 	\textbf{RBI}		&\textbf{PP}	&\textbf{DB}		&\textbf{ DH} \\
\midrule
\textbf{CFG1}				& 11											&0						&0						&0						&0					&0					&0						&0						&0			&0		\\
\textbf{CFG2}				& 7											&0.15					&0.15					&0.15					&0.15				&0.15				&0.15					&0.15					&0.15		&0.15	\\
\textbf{CFG3}				& 5											&0.15					&0.15					&0.15					&0					&0.15				&0						&0.15					&0.15		&0.15	\\
\textbf{CFG4}				& 9											&0						&0.15					&0.15					&0					&0.15				&0.15					&0.15					&0			&0		\\
\textbf{CFG5}				& 8											&0						&0.15					&0.15					&0					&0.15				&0.15					&0						&0			&0		\\
\textbf{CFG6}				& 6 											&0						&0.15					&0.15					&0					&0.15				&0						&0						&0.15		&0.15	\\
\textbf{CFG7}				& 6 											&0.15					&0						&0						&0.15				&0.15				&0.15					&0.15					&0.15		&0.15	\\
\textbf{CFG8}				& 6											&0						&0						&0.15					&0.15				&0.15				&0.15					&0.15					&0.15		&0.15	\\
\textbf{CFG9}				& 7											&0.15					&0.15					&0						&0					&0					&0						&0.15					&0.15		&0.15	\\
\textbf{CFG10}			& 7											&0.15					&0.15					&0						&0					&0					&0						&0.15					&0			&0		\\
\textbf{CFG11}			& 6											&0						&0.15					&0						&0					&0					&0.15					&0.15					&0			&0.15	\\
\textbf{CFG12}			& 6											&0						&0.15					&0						&0					&0					&0.15					&0.15					&0.15		&0		\\
\textbf{CFG13}			& 9											&0						&0						&0						&0					&0					&0						&0.15					&0			&0		\\
\textbf{CFG14}			& 5											&0						&0						&0						&0					&0					&0						&0.15					&0			&0.15	\\
\textbf{CFG15}			& 5											&0						&0						&0						&0					&0					&0						&0.15					&0.15		&0.15	\\
\textbf{CFG16}			& 6											&0						&0.15					&0						&0					&0					&0						&0.15					&0.15		&0.15	\\
\textbf{CFG17}			& 6											&0.15					&0.15					&0.15					&0					&0					&0						&0.15					&0.15		&0.15	\\
\textbf{CFG18}			& 8											&0.15					&0.15					&0.15					&0.15				&0					&0						&0.15					&0.15		&0.15	\\
\textbf{CFG19}			& 7											&0.15					&0.15					&0.15					&0.15				&0.15				&0						&0.15					&0.15		&0.15	\\
\textbf{CFG20}			& 6											&0.15					&0.15					&0.15					&0.15				&0.15				&0.15					&0						&0			&0		\\
\textbf{CFG21}			& 6 											&0.15					&0.15					&0.15					&0.15				&0.15				&0.15					&0						&0			&0.15	\\
\textbf{CFG22}			& 6											&0.15					&0.15					&0						&0					&0					&0						&0.15					&0			&0.15	\\
\textbf{CFG23}			& 4											&0.15					&0.15					&0						&0					&0.15				&0.15					&0.15					&0.15		&0.15	\\
\textbf{CFG24}			& 6 											&0						&0						&0.15					&0.15				&0					&0						&0.15					&0.15		&0.15	\\
\textbf{CFG25}			& 6											&0						&0						&0						&0					&0.15				&0.15					&0.15					&0.15		&0.15	\\
\bottomrule
\end{tabularx}
\normalsize
\label{tab:Carac25CFG}
\end{table}
%%%%%%%%%%%%%%%%%%%%%%%%%%%%%%%%%%%%%%%%%%%%%%%%%%%%%%%%%%%%%%%%%%%%%%%%%%%%%%%%%%%%%%%%%%%%%%%%%%

	
%%%%%%%%%%%%%%%%%%%%%%%%%%%%%%%%%%%%%%%%%%%%%%%%%%%%%%%%%%%%%%%%%%%%%%%%%%%%%%%%%%%%%%%%%%%%%%%%%%
% Table 4 
% JJS : A revoir car:
%  il faut mettre en haut sur l'ensemble "`Instability number"'
% au début ce n'est pas  "`No of ins"' -> configuration
%%%%%%%%%%%%%%%%%%%%%%%%%%%%%%%%%%%%%%%%%%%%%%%%%%%%%%%%%%%%%%%%%%%%%%%%%%%%%%%%%%%%%%%%%%%%%%%%%%
%%%%%%%%%%%%%%%%%%%%%%%%%%
\begin{table}[h!]
\centering
\caption{Frequencies of unstable modes of the $25$ configurations}
\small
\hspace*{-3.5cm}
\begin{tabularx}{20cm}{c*{11}{X}}
\toprule
\multirow{2}{*}{\textbf{Configuration}}								& \multicolumn{11}{c}{\textbf{Instability number}} \\
						&\multicolumn{1}{c}{\textbf{1}}			&\multicolumn{1}{c}{\textbf{2}}	 		&\multicolumn{1}{c}{\textbf{3}}	 		&\multicolumn{1}{c}{\textbf{4}}	 		&\multicolumn{1}{c}{\textbf{5}}	 		&\multicolumn{1}{c}{\textbf{6}}	 		&\multicolumn{1}{c}{\textbf{7}}			&\multicolumn{1}{c}{\textbf{8}}	 		&\multicolumn{1}{c}{\textbf{9}}	 		&\multicolumn{1}{c}{\textbf{10}}	 		&\multicolumn{1}{c}{\textbf{11}}	 	\\ 
 \midrule
\textbf{CFG1} 				&$540.76 $ 			&$675.46$ 			&$1573.4$ 		&$2172.6$ 		&$3441.2$ 		&$3471.2$ 		&$4133.1$ 		&$4244.5$ 		&$4608.8$ 		&$5357.4$ 		&$5769.6$ 		\\
\textbf{CFG2}					&$1976.2$ 			&$3684	$ 				&$3838.9$ 		&$4695.3$ 		&$4944$ 			&$5341.4$ 		&$5934.9$ 		&						&						&						&					\\
\textbf{CFG3}					&$3466.4$ 			&$3860.6$ 			&$4231.8$ 		&$4944.6$ 		&$5273.6$ 		&						&						&						&						&						&					\\
\textbf{CFG4}					&$733.39$ 			&$2994$ 				&$3456.3$ 		&$3892.7$ 		&$4249.3$ 		&$4267.5$ 		&$4630.5$ 		&$4937	$ 			&$5161.1$ 		&						&					\\
\textbf{CFG5}					&$645.31$ 			&$1102.9$ 			&$2543.3$ 		&$3473.7$ 		&$3711.3$ 		&$3897.4$ 		&$4939.6$ 		&$5011.1$ 		&						&						&					\\
\textbf{CFG6}					&$2447.9$ 			&$2544.6$ 			&$3477.2$ 		&$3686.9$ 		&$4121.6$ 		&$4457.3$ 		&						&						&						&						&					\\
\textbf{CFG7}					&$1877.2$ 			&$3432.3$ 			&$3654.5$ 		&$4415.2$ 		&$4692.8$ 		&$5445.7$ 		&						&						&						&						&					\\
\textbf{CFG8}					&$1859.8$ 			&$3491.5$ 			&$3816.3$ 		&$4272	$ 			&$4705$ 			&$4978.3$ 		&						&						&						&						&					\\
\textbf{CFG9}					&$1974.1$ 			&$3105.8$ 			&$3474.6$ 		&$4382.8$ 		&$5140.8$ 		&$5305.2$ 		&$5802.7$ 		&						&						&						&					\\
\textbf{CFG10}				&$713.56$ 			&$2315.4$ 			&$3910.9$ 		&$4415$ 			&$4492.1$ 		&$5244.3$ 		&$5435.3$ 		&						&						&						&					\\
\textbf{CFG11}				&$1876.7$ 			&$3889.7$ 			&$4306	$ 			&$4500.7$ 		&$4609.8$ 		&$4869.1$ 		&						&						&						&						&					\\
\textbf{CFG12}				&$1900.4$ 			&$3809.6$ 			&$4322.5$ 		&$4608$ 			&$4931$ 			&$5863.9$ 		&						&						&						&						&					\\
\textbf{CFG13}				&$698.1$ 				&$2287.2$ 			&$3459.3$ 		&$3645.7$ 		&$3862.1$ 		&$4161	$ 			&$4946.9$ 		&$539.91$ 		&$5434.2$ 		&						&					\\
\textbf{CFG14}				&$3496.8$ 			&$3814.6$ 			&$4120	$ 			&$4733.2$ 		&$5128.4$ 		&						&						&						&						&						&					\\
\textbf{CFG15}				&$1858.7$ 			&$3431$ 				&$3569.8$ 		&$3851.5$ 		&$4108.7$ 		&						&						&						&						&						&					\\
\textbf{CFG16}				&$1913.8$ 			&$3459.3$ 			&$3533.7$ 		&$3810.5$ 		&$4280.3$ 		&$4790.2$ 		&						&						&						&						&					\\
\textbf{CFG17}				&$1964.4$ 			&$3775.5$ 			&$4248.4$ 		&$4357.5$ 		&$4609.9$ 		&$5277.6$ 		&						&						&						&						&					\\
\textbf{CFG18}				&$1963.4$ 			&$2742.6$ 			&$3626.9$ 		&$3800	$ 			&$4656.8$ 		&$4890.6$ 		&$4943	$ 			&$5334.7$ 		&						&						&					\\		
\textbf{CFG19}				&$1975.4$ 			&$3643.2$ 			&$3826.7$ 		&$4691.5$ 		&$4943.2$ 		&$5337.1$ 		&$5925.2$ 		&						&						&						&					\\
\textbf{CFG20}				&$669.19$ 			&$1124.3$ 			&$1882.5$ 		&$4070.7$ 		&$4653.5$ 		&$5021.4$ 		&						&						&						&						&					\\
\textbf{CFG21}				&$1881.4$ 			&$2546.7$ 			&$2560.6$ 		&$2757.1$ 		&$3677	$ 			&$4668.1$ 		&						&						&						&						&					\\
\textbf{CFG22}				&$3407.6$ 			&$3906.4$ 			&$4361.3$ 		&$4615.3$ 		&$4941.6$ 		&$5254.2$ 		&						&						&						&						&					\\
\textbf{CFG23}				&$1970.3$ 			&$3800.5$ 			&$4552.6$ 		&$4929.3$ 		&						&						&						&						&						&						&					\\
\textbf{CFG24}				&$854.4$ 				&$1851.8$ 			&$3593.2$ 		&$3759.1$ 		&$4305.3$ 		&$4689.9$ 		&						&						&						&						&					\\
\textbf{CFG25}				&$1859.8$ 			&$3491.5$ 			&$3816.3$ 		&$4272	$ 			&$4705	$ 			&$4978.3$ 		&						&						&						&						&					\\
\bottomrule
\end{tabularx}\hspace*{-2cm}
\normalsize
\label{tab:FreqIns25CFG}
\end{table}
%%%%%%%%%%%%%%%%%%%%%%%%%%


%%%%%%%%%%%%%%%%%%%%%%%%%%
\begin{figure}[h!]
\centering
\includegraphics[height=0.75\textheight]{MAC_ModesIns_25CFG.eps}
\caption{AutoMAC of the 25 configurations - Red dot lines corresponds to the separation between the configurations}
\label{fig:MAC25CFG}
\end{figure}
%%%%%%%%%%%%%%%%%%%%%%%%%%

%%%%%%%%%%%%%%%%%%%%%%%%%%
\begin{figure}[h!]
\centering
\includegraphics[height=0.75\textheight]{FMAC_ModesIns_25CFG.eps}
\caption{FAutoMAC of the 25 configurations - MAC values with a frequency scale}
\label{fig:FMAC25CFG}
\end{figure}
%%%%%%%%%%%%%%%%%%%%%%%%%%


%%%%%%%%%%%%%%%%%%%%%%%%%%
\begin{figure}[tb]
%\begin{changemargin}{-4cm}{-4cm}
	\centering
	\begin{tabular}{@{}cc@{}}
	\subfloat[a][]{
	\includegraphics[height=.3\textheight]{MAC_ModesIns_12CFG_PtsDisc.eps}
	\label{fig:MAC_disc}}&
	\subfloat[b][]{
	\includegraphics[height=.3\textheight]{MAC_ModesIns_12CFG_Etrier.eps}
	\label{fig:MAC_etrier}}\\
	\subfloat[c][]{
	\includegraphics[height=.3\textheight]{MAC_ModesIns_12CFG_Chape.eps}
	\label{fig:MAC_chape}} &
	\subfloat[d][]{
	\includegraphics[height=.3\textheight]{MAC_ModesIns_12CFG_PlInt.eps}
	\label{fig:MAC_plint}}\\
	\subfloat[e][]{
	\includegraphics[height=.3\textheight]{MAC_ModesIns_12CFG_PlExt.eps}
	\label{fig:MAC_plext}}
	\end{tabular}
%	\end{changemargin}
	\caption{AutoMAC matrices of the unstable modes of $12$ configurations for the different components: (a) Disc, (b) Caliper, (c) Bracket, (d) Inner Pad, (e) Outer Pad}
	\label{fig:MACElem}
\end{figure}
%%%%%%%%%%%%%%%%%%%%%%%%%%

%%%%%%%%%%%%%%%%%%%%%%%%%%%%%%%%%%%%%%%%%%%%%%%%%%%%%%%%%%%%%%%%%%%%%%%%%%%%%%%%%%%%%%%%%%%%%%%%%%
% Table 5
% JJS : pourquoi les configurations ne s'appellent plus CFG1 CFG5 CFG14 et CFG3 
% les mettre dans un ordre chronologique (à moins qu'il y ait une autre logique que je n'ai pas vue...
% est-ce normal d'avoir une seule valeur pour 2 configurations donnée (exactement la même valeur . )
% au début ce n'est pas  "`N$^o$ of the configuration"' -> configuration
%%%%%%%%%%%%%%%%%%%%%%%%%%%%%%%%%%%%%%%%%%%%%%%%%%%%%%%%%%%%%%%%%%%%%%%%%%%%%%%%%%%%%%%%%%%%%%%%%%
%%%%%%%%%%%%%%%%%%%%%%%%%%
\begin{table}[h!]
\centering
\caption{Characteristics of observed mode shapes}
\begin{tabularx}{\hsize}{lYYYY}
\toprule
\textbf{Configuration}   	 				& \textbf{CFG3} 	& \textbf{CFG4}  	& \textbf{CFG5} 		& \textbf{CFG14}  \\
\cmidrule(lr){2-3}\cmidrule(lr){4-5}
\textbf{Freq. (Hz)}                  			& $4231.8$ 		& $4267.5$ 	 		&  $3473.7$      		& $3496.8$  		\\
\textbf{Real Part}                   			& $1429$			& $1376.7$       		& $627.63 $      		& $302.73$		 \\
\textbf{MAC Global}  						&  \multicolumn{2}{c}{$0.371$}   		&  \multicolumn{2}{c}{$0.575$}  		\\
\textbf{MAC Disc}  							&  \multicolumn{2}{c}{$0.944$}  		&  \multicolumn{2}{c}{$0.955$}  		 \\
\textbf{MAC Bracket}  					&  \multicolumn{2}{c}{$0.528$}  		&  \multicolumn{2}{c}{$0.053$}  	 	\\
\textbf{MAC Caliper}  						&  \multicolumn{2}{c}{$0.042$}  		&  \multicolumn{2}{c}{$0.088$}  	 	\\
\textbf{MAC Inner Pad}  					&  \multicolumn{2}{c}{$0.222$}  		&  \multicolumn{2}{c}{$0.126$}  	 	\\
\textbf{MAC Outer Pad}  				&  \multicolumn{2}{c}{$0.566$}  		&  \multicolumn{2}{c}{$0.203$}  		\\
\bottomrule
\end{tabularx}
\label{tab:ModesComparesCarac}
\end{table}
%%%%%%%%%%%%%%%%%%%%%%%%%%

%%%%%%%%%%%%%%%%%%%%%%%%%%
\begin{figure}[tb]
%\begin{changemargin}{-4cm}{-4cm}
	\centering
	\begin{tabular}{@{}cc@{}}
	\subfloat[a][]{
	\includegraphics[height=.3\textheight]{CFG5_VueFace_acCarac.eps}
	\label{fig:CFG5Face}}&
	\subfloat[b][]{
	\includegraphics[height=.3\textheight]{CFG14_VueFace_acCarac.eps}
	\label{fig:CFG14Face}}\\
	\subfloat[c][]{
	\includegraphics[height=.3\textheight]{CFG5_VueProfil_acCarac.eps}
	\label{fig:CFG5Profile}}&
	\subfloat[d][]{
	\includegraphics[height=.3\textheight]{CFG14_VueProfil_acCarac.eps}
	\label{fig:CFG14Profile}}\\
	\end{tabular}
%	\end{changemargin}
	\caption{Comparison of mode shapes of configurations 5 (left) and 14 (right) - Face view (top) and Profile view (bottom)}
	\label{fig:CompCFG5_14}
\end{figure}
%%%%%%%%%%%%%%%%%%%%%%%%%%

%%%%%%%%%%%%%%%%%%%%%%%%%%
\begin{figure}[tb]
%\begin{changemargin}{-4cm}{-4cm}
	\centering
	\begin{tabular}{@{}cc@{}}
	\subfloat[a][]{
	\includegraphics[height=.35\textheight]{CFG3_VueFace.eps}
	\label{fig:CFG3Face}}&
	\subfloat[b][]{
	\includegraphics[height=.35\textheight]{CFG4_VueFace.eps}
	\label{fig:CFG4Face}}\\
	\subfloat[c][]{
	\includegraphics[height=.35\textheight]{CFG3_VueProfil.eps}
	\label{fig:CFG3Profile}}&
	\subfloat[d][]{
	\includegraphics[height=.35\textheight]{CFG4_VueProfil.eps}
	\label{fig:CFG4Profile}}\\
	\end{tabular}
%	\end{changemargin}
	\caption{Comparison of mode shapes of configurations 3 (left) and 4 (right) - Face view (top) and Profile view (bottom)}
	\label{fig:CompCFG3_4}
\end{figure}
%%%%%%%%%%%%%%%%%%%%%%%%%%

%%%%%%%%%%%%%%%%%%%%%%%%%%%%%%%%%%%%%%%%%%%%%%%%%%%
\subsection{Influence of the contact conditions per frequency range and restriction on the concept of the most unstable modes}
\label{sec:}
%%%%%%%%%%%%%%%%%%%%%%%%%%%%%%%%%%%%%%%%%%%%%%%%%%%

 As it was previously shown, it is not possible to understand the role of each internal contact interfaces on the squeal propensity by tracking the unstable modes for all the frequency range of interest. In other words,  being able to establish a relevant link between the appearance of instabilities via the mode shapes of instabilities and certain conditions of internal contacts is not easily achievable. So another strategy has to be adopted to analyse the problem under study and to better understand the role of each internal interface on the appearance of instabilities. In order to achieve such an objective, the automotive brake system is now studied for some specific frequency range of interest.

To illustrate the proposed approach, the occurrences of unstable frequencies are considered and displayed in Figure~\ref{fig:OccFreqIns}. Twelve small frequency intervals that focused on a peak of occurrences are retained, the main purpose being to establish whether some specific internal contact conditions are observable for a given peak of unstable frequencies. The characteristics of each frequency range are given in Table~\ref{tab:FreqOcc}. Then, for each interval, the repartition of the contact conditions of the nine internal interfaces associated to the unstable frequencies are determined. Results are summarized in Figure~\ref{fig:RepartitionEtatContact} where, for each frequency range and for each internal contact, the percentage of repartition between a sliding state (in blue) and a frictional state (in yellow) is given. 
For a few cases, the role of some internal contact states in the existence of the instability of a frequency range of interest is obvious. For example, under $1140$ Hz, the existence of instabilities is always related to a sliding state of the contacts DB and DH. For the frequency range  $[1400;1470]$ Hz, instabilities are mainly related to a frictional state of contacts DB, PP and RHI and a sliding state of contacts DH and RHE. For $[1150;1650]$ Hz, the emergence of instabilities corresponds to a sliding state of contacts RBE and RCBE. However, if such an analysis gives some interesting tendencies in some cases, there is no concrete conclusion for most frequency ranges. See for instance the frequency ranges $[1800;2000]$ Hz and $[3775;3930]$ Hz where the repartition is around 50\% for each internal contact interface. 

This approach is interesting but seems limited since it is possible to find two completely different mode shapes with similar natural frequencies. Moreover, the results are strongly influenced by the choice of the frequency range, which can be difficult to determine when the peaks of the occurrences of unstable frequencies are not clearly distinct. For this reason a more comprehensive approach must be defined. Instead of characterizing the condition of instabilities existence, the choice is made to characterize the most unstable configuration of larger frequency ranges, which can be assume at a first glance as the worst case scenario for an automotive brake system. This choice can be interpreted as a simplification or a restriction of the general problem to the study of the influence of internal contact for the most unstable modes.

This leads us to identify the contact state of each of the nine internal interface that bring the system in the most unstable configuration. The notion of the "most unstable configuration" is not necessarily unique and universal. Here, the most unstable configuration is defined by the configuration with the maximum real part (i.e. the highest growth rate of the unstable mode). The frequency range under study is reduced to $[0;6000]$ Hz in order to be concise in our analysis, the idea being to illustrate the relevance of the proposed methodology. 

Then the methodological approach is the following one: the status of one internal contact interface is imposed (sliding state or frictional state) while all other internal interfaces can be in any state. Then $2^8=256$ CEA are performed and the maximum of real parts is determined. This procedure is repeated for each of the nine internal contact interfaces by applying the sliding state (with $\mu_{intern} =0$) or friction (with $\mu_{intern} = 0.15$). Because some configurations are redundant, only 512 CEA calculations are required. The results obtained with $\mu_{pad-disc}=0.3$ are presented in Figure~\ref{fig:ContactMaxRe}.

A relevant analysis of the impact of each internal contact is then possible. For example, if the contact between the piston and the inner pad is sliding (see Figure~\ref{fig:ContactTtesFreq} at contact PP and blue point), the maximum real part is equal to 725, while it is equal to 1100 in the case of frictional state for PP is (see yellow point). Here the "most unstable configuration" versus the contact PP corresponds to the frictional state. More generally, it also appears that the two internal contacts that offer the most variation for the "most unstable configuration" are  the piston-inner pad contact (PP) and the inner top radial contact (RHI). Moreover, the  influences of both the bottom and top caliper-pad contact (DB and DH) appear to be less important. 

These results indicates that considering the possibility of a change of contact status for the two internal contacts DB and DH is not necessary for a future analyse and design studies. Moreover according to  the level of accuracy required in a design process, it is also possible to define the internal contacts whose two states (sliding and friction) to be kept and those for whom a contact state is privileged. For example, if the most unstable scenario has to be characterized at about 5\%, then it is characterised by a frictional state for internal contacts PP, RHE and RBI and by a sliding state for internal contacts RHI, RCBI, RCBE and RBE. The state of the two internal contact DB and DH is in the margin of error. 

Now the same analyse can be carried out for different frequency ranges in order to better understand the role of each internal interface on the appearance of instabilities by frequency ranges of interest. Results are displayed in Figure~\ref{fig:ContactParFreq}. As in the previous case, it is possible to create a hierarchy between the different internal contacts and to decide whether or not to retain the possibility of the two state for a specific internal contact in a design process. It is also observed that the internal contacts that appear to be the most influential on the "most unstable configuration" depend of the considered frequency range. For instance, the contribution of the contact PP is low for $[950;2005]$ Hz but crucial for $[4000;4890]$ Hz. On the contrary, the contribution of the two contacts DB and DH is high for the two frequency ranges  $[0;950]$ Hz and $[950;2005]$ Hz but low between $[4000;4890]$ Hz. It is also noted that the influence of the internal contact states DB and DH  versus the "most unstable configuration" are reversed on the first two frequency ranges. For the first frequency range $[0; 950]$ Hz the "most unstable configuration" is obtained for the sliding configuration of the internal contacts DB and DH, whereas it corresponds to the friction state of DB and DH for the second frequency range $[950;2005]$~Hz.

In conclusion, the proposed strategy allows us to rule on the effect of all independent internal contacts. Some internal contacts have a large contribution and play a key role on the propensity of automotive brake squeal, whereas the others have a small impact and only add a slight variation on the instability phenomena. Moreover it is possible to propose a restriction of the potential evolution of the contact states to be considered for each internal interface according to a given design process versus a frequency range of interest. Considering that the CEA enables to get the squeal propensity but has the disadvantages of being under- or over- predictive, the determination of this area and its characteristics may be enough in a design process. 

%%%%%%%%%%%%%%%%%%%%%%%%%%
\begin{figure}[h!]
\hspace{-3.5cm}
%\centering
\includegraphics[height=0.55\textheight]{Occurrences_freq_instables.eps}
\caption{Occurrences of unstable frequencies and intervals of study}
\label{fig:OccFreqIns}
\end{figure}
%%%%%%%%%%%%%%%%%%%%%%%%%%


%%%%%%%%%%%%%%%%%%%%%%%%%%%%%%%%%%%%%%%%%%%%%%%%%%%%%%%%%%%%%%%%%%%%%%%%%%%%%%%%%%%%%%%%%%%%%%%%%%
% Tables 6 et 7
% mettre plutot [aaa ; bbb] cela sera plus clair et classiquemenr un tableau donnant des intervalles de freq est mis en colonne (et non ligne, soit dans l'autre sens)
%%%%%%%%%%%%%%%%%%%%%%%%%%%%%%%%%%%%%%%%%%%%%%%%%%%%%%%%%%%%%%%%%%%%%%%%%%%%%%%%%%%%%%%%%%%%%%%%%%
%%%%%%%%%%%%%%%%%%%%%%%%%%
\begin{table}[h!]
\centering
\caption{Frequencies intervals}
\begin{tabular}{cc}
\toprule
\textbf{N$^o$}   	 	& 		\textbf{Freq interval}  	\\
\midrule
$\textbf{1}$			&	[500; 800]						\\
$\textbf{2}$			&  [1070; 1140]  				\\
$\textbf{3}$ 			&  [1400; 1470]  		 		\\
$\textbf{4}$			&  [1550; 1650] 		 			\\
$\textbf{5}$			&  [1800; 2000]  				\\
$\textbf{6}$  			&  [2970; 3020]  				\\
$\textbf{7}$  			& [3385; 3515] 					\\
$\textbf{8}$  			& [3775; 3930]  					\\
$\textbf{9}$  			& [4200; 4320] 					\\
$\textbf{10}$ 			& [4880; 4970]					 \\
$\textbf{11}$ 			& [4988; 5055] 					\\
$\textbf{12}$ 			& [5400; 5595] 					\\
\bottomrule
\end{tabular}
\label{tab:FreqOcc}
\end{table}
%%%%%%%%%%%%%%%%%%%%%%%%%%


%%%%%%%%%%%%%%%%%%%%%%%%%%
\begin{figure}[h!]
\hspace{-3.5cm}
%\centering
\includegraphics[height=0.6\textheight]{Repartition_Contacts_ParFreq.eps}
\caption{Percentage of repartition of contact states for the different frequency intervals - Blue: sliding state ($\mu_{intern} = 0$) and yellow: frictional state ($\mu_{intern} = 0.15$)}
\label{fig:RepartitionEtatContact}
\end{figure}
%%%%%%%%%%%%%%%%%%%%%%%%%%


%%%%%%%%%%%%%%%%%%%%%%%%%%
\begin{figure}[tb]
	\centering
	\begin{tabular}{@{}c@{}}
		\subfloat[a][]{
			\includegraphics[height=.4\textheight]{MaxPartieReelle_AllFreq_MU03.eps}
			\label{fig:ContactMaxRe_MU03}} \\	
		\subfloat[b][]{
			\includegraphics[height=.4\textheight]{MaxPartieReelle_AllFreq.eps}
			\label{fig:ContactMaxRe_MU05}} \\
	\end{tabular}	
	\caption{Maximum of real part assuming a contact condition - Blue for a sliding state ($\mu_{intern} = 0$) and yellow for a frictional state ($\mu_{intern}=0.15$) - for $\mu_{pad/disc} = 0.3$ (top) and $\mu_{pad/disc} = 0.5$ (bottom)}
	\label{fig:ContactMaxRe}
\end{figure}
%%%%%%%%%%%%%%%%%%%%%%%%%%


%%%%%%%%%%%%%%%%%%%%%%%%%%
\begin{figure}[tb]
	\vspace*{-4cm}
	\begin{tabularx}{20cm}{Y}
	\subfloat[a][]{	\hspace*{-6cm}
		\includegraphics[height=.6\textheight]{MaxPartieReelle_ParFreq_MU03.eps}
		\label{fig:ContactParFreq_MU03}} \\
	\subfloat[b][]{	\hspace*{-6cm}
		\includegraphics[height=.6\textheight]{MaxPartieReelle_ParFreq.eps}
		\label{fig:ContactParFreq_MU05}} \\
	\end{tabularx}
	\caption{Maximum of real part assuming a contact condition per frequency range - Blue for a sliding state ($\mu_{intern} = 0$) and yellow for a frictional state ($\mu_{intern}=0.15$) - for $\mu_{pad/disc} = 0.3$ (top) and $\mu_{pad/disc} = 0.5$ (bottom)}
	\label{fig:ContactParFreq}
\end{figure}
%%%%%%%%%%%%%%%%%%%%%%%%%%



%%%%%%%%%%%%%%%%%%%%%%%%%%%%%%%%%%%%%%%%%%%%%%%%%%%
\section{Use of a genetic algorithm to determine the most critical scenario}
\label{subsec:}
%%%%%%%%%%%%%%%%%%%%%%%%%%%%%%%%%%%%%%%%%%%%%%%%%%%

%%%%%%%%%%%%%%%%%%%%%%%%%%%%%%%%%%%%%%%%%%%%%%%%%%%
\subsection{Motivation for an industrial point of view}
%%%%%%%%%%%%%%%%%%%%%%%%%%%%%%%%%%%%%%%%%%%%%%%%%%%

This section is devoted to the use of a Genetic Algorithm (GA) to identify the case scenario defined as "the most unstable configuration". Finding this type of configuration is indeed an objective of interest for the PSA automotive industry. This may seem paradoxical since one would expect to look for the design leading to the least squeaky system. However, the prohibitive cost of calculation makes it impossible to propose a global design strategy that includes the internal interfaces. So the proposed procedure is divided into two steps. First, because the actual behaviour of these internal contact interfaces in service is uncertain and not controlled, the most harmful design in regard to them is determined. And finally, a design study on controllable parameters is performed on this "most unstable configuration" in order to mitigate it. For example, the selected instability in the automotive brake system design should then be eliminated or reduced by changing the geometry or material properties of the brake components to decouple the coupling modes.

To cope with the high number of internal contacts and the too high numerical cost of taking all configurations into account, a first reduction of the problem size is realized by only assuming two different contact states  for each internal interface: sliding with $\mu_{intern} =0$ or friction with $\mu_{intern} \neq 0$, and the variation of the coefficient of friction $\mu_{intern}$ for $\mu_{intern} \neq 0$ is not taken into account (this assumption is in agreement with the first conclusions of the study). However, it still represents a high number of CEA to perform since there are numerous internal contacts, so it is necessary to find a strategy to overcome this numerical cost. Because the problem is discontinuous, a strategy based on GA is undertaken.


%%%%%%%%%%%%%%%%%%%%%%%%%%%%%%%%%%%%%%%%%%%%%%%%%%%%
\subsection{Genetic algorithm and adaptation for the problem under study }
\label{subsec:GA}
%%%%%%%%%%%%%%%%%%%%%%%%%%%%%%%%%%%%%%%%%%%%%%%%%%%%

This section aims to briefly recall the concept and approach associated with Genetic algorithm (GA) that is a kind of Evolutionary Algorithm (EA) used to approximate the solution of a given problem in a reasonable time. It is about evolving a population in which the individuals try to improve themselves, generation after generation with respect to the objectives. In this context, the vocabulary to describe the problem comes from biology: 
\begin{itemize}
	\item an \textit{individual} is a solution of the problem and is described by its genes,
	\item a \textit{population} is a group of individuals,
	\item a \textit{generation} is an iteration of the algorithm and corresponds to the current population.
\end{itemize}

Thus, a GA only requires a genetic description of the problem and a function to evaluate the solution domain. In the study presented here, the genes of individuals correspond to the status of the different internal contacts. $6$ objective functions are considered and correspond to the maximal real part for each frequency range provided in \ref{tab:DecoupeFreqAlgo}. The general methodology is sketched in Figure \ref{fig:SchemaAlgo}(a). The different steps are the following:
\begin{itemize}
	\item Step 1: Initialisation with an initial population created randomly,
	\item Step 2: Evaluation of the fitness function for each individual,
	\item Step 3: Creation of new individuals by selection, crossover and mutations,
	\item Step 4: Evaluation of new individuals,
	\item Step 5: Creation of a new population with old and new individuals.
\end{itemize}

Steps from 3 to 5 are iterated until a stop criterion is reached.

The different operators that characterized the GA are :
\begin{itemize}
	\item Selection: different strategies are proposed in the literature  as the wheel selection, the tournament selection or even the elitist selection. In the present study, the tournament selection is used. It consists in the creation of tournaments between individuals chosen randomly in the population and the best of them are selected.
	\item Crossover: this operator is used to create the genes of the new generation from those of the previous one. Several techniques are possible as the single-point crossover, the two-points crossover or the uniform crossover. Here, a uniform crossover is considered. For each gene, the child's gene is chosen randomly between one of the two parents.
	\item Mutation: this operator creates random modifications of genes. This is important for GA to replace "forgotten" genes of old population or to test new combinations.
\end{itemize}

For more details, the reader can refer to \cite{Deb2002Fast,whitley1994GA}. GA are adapted to both Single-Objective (SO) optimisation and Multi-Objective (MO) optimisation and according to the application, different methods exist. For the present study, the objective is to determine the internal contact status that give the maximum real part on each frequency range of interest. Since there are $6$ frequency ranges an so $6$ objective functions, a MO algorithm may be conceivable. However, a MO-GA determines the Pareto front of the different objective functions, which is not what we are looking for in our study. In fact, our goal here is to determine the solution (i.e. "the most unstable configuration") for each frequency range. So the strategy consists to launch a SO-GA for each function. If they started one after the other, the different populations cannot learn from each other. Also the proposition developed here relies in the launch of the six SO-GA simultaneously with a common population. The general sketch of the method is given in Figure~\ref{fig:SchemaAlgo}(b). Overall operation remains the same: crossover, mutations and child evaluations are realized for the global population. The difference relies in the selection of parents: all the population is considered and the $N/N_f$ best individuals for each objective functions are selected for the next step, where $N$ is the number of parents to select and $N_f$ the number of objective functions. In order to limit the number of CEA to perform, an archive is created with all tested individuals and new individuals must not be inside.



%%%%%%%%%%%%%%%%%%%%%%%%%%
\begin{figure}[tb]
	\hspace{-3.5cm}
	\begin{tabular}{@{}cc@{}}
	\subfloat[a][]{
	\includegraphics[height=.4\textheight]{PlanComplexeAvecDecoupeFreq_MU03.eps}
	\label{fig:PlanCplxDecoupeFreq_MU03}}&
	\subfloat[b][]{
	\includegraphics[height=.4\textheight]{PlanComplexeAvecDecoupeFreq.eps}
	\label{fig:PlanCplxDecoupeFreq_MU05}}\\
	\end{tabular}
	\caption{Eigenvalues in the complex plan and frequency ranges for $\mu_{pad/disc} = 0.3$ (left) and $\mu_{pad/disc} = 0.5$ (right)}
	\label{fig:PlanCplxDecoupeFreq}
\end{figure}
%%%%%%%%%%%%%%%%%%%%%%%%%%


%%%%%%%%%%%%%%%%%%%%%%%%%%
\begin{table}[h!]
\centering
\caption{Frequencies intervals}
\begin{tabular}{cc}
\toprule
\textbf{N$^\text{o}$}   	 	&  	\textbf{Freq. interval} \\
\midrule
1   &  [950; 2005] \\
2  &  [2005; 2881] \\
3  &  [2881; 4000] \\
4  & [4000; 4890] \\
5 & [4890; 6000] \\
\bottomrule
\end{tabular}
\label{tab:DecoupeFreqAlgo}
\end{table}
%%%%%%%%%%%%%%%%%%%%%%%%%%


%%%%%%%%%%%%%%%%%%%%%%%%%%%%%%%%%%%%%%%%%%%%
\begin{figure}[tb]
\vspace{-4.5cm}
	\begin{tabular}{@{}c@{}}
	\subfloat[a][]{
		\tikzstyle{Algo1}=[draw, rounded corners = 3pt, thick, minimum height = 1cm, text width =3cm,text centered]
		\begin{tikzpicture}[scale=0.5]
			\node[Algo1,minimum height=1.2cm] (P0) at (0,0) {Initialisation Population P0} ;
			\node[Algo1, below=15pt of P0] (P1) {Evaluation} ;
			\node[Algo1, below right=35pt and -5pt of P1] (D1){Selection} ;
			\node[Algo1, minimum height=1.2cm, below = 15pt of D1](D2){Crossover Mutations};
			\node[Algo1, below=15pt of D2](D3){Evaluation of new individuals};
			\node[Algo1, below=15pt of D3](D4){New population};
			\node[draw, ellipse, thick, text width =1.5cm,text centered, below=220pt of P1] (P3) {Stop criterion} ;
			\node[Algo1, below left=90pt and -5pt of P1] (G1) {New generation} ;
			\node[below=5pt of P1] (Ref1) {};
			\node[above=5pt of P3] (Ref2) {};
			\node[below=30pt of P3] (Ref3) {};
			\draw[-latex,thick] (P0) -- (P1);
			\draw[-latex,thick] (D1) -- (D2);
			\draw[-latex,thick] (D2) -- (D3);
			\draw[-latex,thick] (D3) -- (D4);
			\draw[-latex,thick] (P1) -- (Ref1.center);
			\path[linearrow,thick] (Ref1.center) -| (D1.north);
			\draw(G1.north) |- (Ref1.center);
			\draw[-latex,thick] (D4.south) |- (P3.east);
			\draw[-latex,thick] (P3.west) -- ++ (-20pt,0pt) node[above] {No} -| (G1.south);
			\draw[-latex,thick] (P3.south) -- node[midway,right] {Yes} (Ref3);
		\end{tikzpicture}
	\label{fig:AlgoGene}}\\
	\subfloat[b][]{
		\tikzstyle{Algo1}=[draw, rounded corners = 3pt, thick, minimum height = 1cm, text width =3cm,text centered]
	\begin{tikzpicture}[scale=0.5]
		\node[Algo1,minimum height=1.2cm] (P0) at (-2,0) {Initialisation Population P0} ;
		\node[Algo1, below=15pt of P0] (P1) {Evaluation} ;
		\node[Algo1, below right=35pt and -5pt of P1] (D1){Selection of best individuals for each function} ;
		\node[Algo1, minimum height=1.2cm, below = 15pt of D1](D2){Crossover Mutations};
		\node[Algo1, below=15pt of D2](D3){Evaluation of new individuals};
		\node[Algo1, below=15pt of D3](D4){New population};
		\node[draw, ellipse, thick, text width =1.5cm,text centered, below=220pt of P1] (P3) {Stop criterion} ;
		\node[Algo1, below left=90pt and -5pt of P1] (G1) {New generation} ;
		\node[Algo1, fill=gray!20,below right=-10pt and 75pt of P1] (Archive) {Archive};
		\node[below=5pt of P1] (Ref1) {};
		\node[above=5pt of P3] (Ref2) {};
		\node[below=30pt of P3] (Ref3) {};
		\draw[-latex,thick] (P0) -- (P1);
		\draw[-latex,thick] (D1) -- (D2);
		\draw[-latex,thick] (D2) -- (D3);
		\draw[-latex,thick] (D3) -- (D4);
		\draw[-latex,thick] (P1) -- (Ref1.center);
		\path[linearrow,thick] (Ref1.center) -| (D1.north);
		\draw(G1.north) |- (Ref1.center);
		\draw[-latex,thick] (D4.south) |- (P3.east);
		\draw[-latex,thick] (P3.west) -- ++ (-20pt,0pt) node[above] {No} -| (G1.south);
		\draw[-latex,thick] (P3.south) -- node[midway,right] {Yes} (Ref3);
		\draw[-latex, line width=1.2pt,dashed] (P0.east) -- ++ (50pt,0pt) node[above]{Save} -| (Archive);
		\draw[-latex,line width=1.2pt,dashed] (D4) -| node[near end,right]{Save} (Archive.335);
		\draw[-latex,dotted,line width=2pt] (Archive.220) -- ++(0pt,-47pt) node[right] {Compare} |- node [above] {}(D2.east);
	\end{tikzpicture}
	\label{fig:AlgoMulitGene}}\\
	\end{tabular}
	\caption{Diagrams of algorithms: genetic algorithm (left) and multi-population genetic algorithm (right)}
	\label{fig:SchemaAlgo}
\end{figure}
%%%%%%%%%%%%%%%%%%%%%%%%%%%%%%%%%%%%%%%%%%%%%



%%%%%%%%%%%%%%%%%%%%%%%%%%%%%%%%%%%%%%%%%%%%%%%%%%%%%%
\subsection{Numerical results}
\label{subsec:}
%%%%%%%%%%%%%%%%%%%%%%%%%%%%%%%%%%%%%%%%%%%%%%%%%%%%%%%

Two different sets of parameters are considered for the GA. They are given in Table~\ref{tab:CaracAlgo}. The first set is characterized by a smaller population and a low number of potential parents but more generations are accepted, when the second set has a bigger population and a higher number of potential parents but the maximal number of generations is smaller. To compare the performances of the two sets, each algorithm is launched $1000$ times. 

An example of obtained results is displayed in Figure~\ref{fig:AlgoG_PlanCplx} for both set of parameters. The different individuals tested by the algorithm are in black, the solution obtained by the algorithm in blue and the theoretical solution in red. It is clear that the algorithm gives a good approximation of the reference solution and the map created by the different individuals gives a good overview of the complete complex plan (compare to Figure~\ref{fig:PlanCplxDecoupeFreq}).

To evaluate the robustness of the algorithm, the percentage of cases where the correct solution is estimated in a certain margin of error is displayed for each function (see Figure~\ref{fig:AlgoG_Precision}). In both cases, solutions are well approximated except in the case of the third frequency range. \tb{This latter case is particular. Indeed considering reference results, if one considers the second eigenvalue closest to the solution for this frequency range, then there is $22$\% of difference between the real parts. For the other frequency ranges, the algorithm proves its efficiency and gives a satisfactory estimation. If a margin of error of 10\% is considered, the second set of parameters has better results than the first one. For example, between $[950;2005]$ Hz, with the first set the solution is estimated with less than 10\% of error in 75\% of cases, whereas it is in 90\% of cases for the second set of parameters. However, if a margin of error of 20\% is considered, then their performances are very similar.}

The repartition of the contact status of GA solutions is also displayed in Figure~\ref{fig:AlgoG_Repartition}. Results are very similar for both sets of parameters and some tendencies appear. For example, for the second frequency range (between $[950;2005]$ Hz), GA solutions always have a frictional state for contacts DB, DH and almost RBI; and always a sliding state for contacts RCBE and almost RBE. These results have to be compared with results presented in Figure~\ref{fig:ContactParFreq}. For the second frequency range, the most influential contacts are DB, DH, RCBE and RBE. It appears here that the GA has identified quickly the most influential contacts on the frequency range. In the same way, similar analysis can be made on the other frequency ranges for different contacts.

From a computational cost point of view, the second set of parameters requires $36$ extra calculations compared to the first set. These extra calculations bring a better precision on the estimation of the real part; however, they do not bring real contribution on the identification of the most influential contacts. In fact, the main influential contacts are determined in a few generations, but the accurate estimation of the maximum real part has an important cost: $36$ extra calculations improve the estimation at 10\% of the maximum real part of about 10\%.
\textcolor{red}{ je ne comprends pas la phrase du haut tu veux dire : $36$ extra calculations improve the estimation of the maximum real part of about 10\%.}
\tb{en fait j'aimerais dire qu'avec 36 CEA de plus, sur la figure 19, le baton bleu foncé monte de 10\%, et ce baton bleu foncé correspond à l'estimation à 10\% près de la partie réelle. L'idée à faire dans le fond c'est que l'estimation précise de la partie réelle coûte cher (et je cherchais à illustrer ça) mais que l'identification des contacts les plus influents coûte bcp moins chère }

In conclusion the use of a GA enables to identify with a few CEA the area with the maximum real part and the most influential contact states are quickly identified for different frequency range on interest. Actually, it is the accurate determination of the maximum real part that requires a higher number of computations. In a context were the case scenario defined by "the most unstable configuration" could be determined with a margin of error, the GA gives satisfactory results. The characteristics of the most unstable configuration are estimated by running only $ 86 $ CEA instead of the $ 512 $  calculations to be done without GA. So this strategy appears to be of interest for an industrial point of view in order to find the most harmful design in regard to the internal contact interfaces while saving in numerical computing time.


%%%%%%%%%%%%%%%%%%%%%%%%%%
\begin{table}[h!]
	\hspace{-4cm}
	\caption{Genetic algorithm parameters}
	\begin{tabularx}{16cm}{c*{2}{Y}}
		\toprule
		\textbf{N$^o$}   	 									& $\textbf{1}$ 	& $\textbf{2}$  	\\
		\midrule
		\textbf{Population size}   						&  $25$      		& $30$  			\\
		\textbf{Number of selected individuals}		&  $12$				& $20$				\\
		\textbf{Number of generations}    				&  $5$     			& $4$				\\
		\textbf{Average number of calculations} 	&  $ 86.3$			& $122.8$			\\
		\bottomrule
	\end{tabularx}
\label{tab:CaracAlgo}
\end{table}
%%%%%%%%%%%%%%%%%%%%%%%%%%

%%%%%%%%%%%%%%%%%%%%%%%%%%
\begin{figure}[tb]
	\begin{tabular}{c}
	\subfloat[a][]{
		\includegraphics[height=.45\textheight]{AlgoGene_PlanComplexe_Carac2.eps}
		\label{fig:AlgoGene_PlanCplx_Carac2}}\\
	\subfloat[b][]{
		\includegraphics[height=.45\textheight]{AlgoGene_PlanComplexe_Carac4.eps}
		\label{fig:AlgoGene_PlanCplx_Carac4}}\\
	\end{tabular}
	\caption{Eigenvalues computed during the algorithm with the first set of parameters (top) and the second (bottom): black: eigenvalues computed during the algorithm, red stars: theoretical solutions, red points: eigenvalues of the theoretical solutions, blue circle: solutions of the algorithm}
	\label{fig:AlgoG_PlanCplx}
\end{figure}
%%%%%%%%%%%%%%%%%%%%%%%%%%

%%%%%%%%%%%%%%%%%%%%%%%%%%
\begin{figure}[tb]
	\begin{tabular}{c}
	\subfloat[a][]{
		\includegraphics[height=.45\textheight]{AlgoGene_Precision_Carac2.eps}
		\label{fig:AlgoGene_Precision_Carac2}}\\
	\subfloat[b][]{
		\includegraphics[height=.45\textheight]{AlgoGene_Precision_Carac4.eps}
		\label{fig:AlgoGene_Precision_Carac4}}\\
	\end{tabular}
	\caption{Precision of the solution obtained for $1000$ initialisations for the first set of parameters (top) and the second (bottom) on each frequency range - dark blue: 10\% of error, blue: 20\% of error, green: 25\% of error and yellow: 30\% of error}
	\label{fig:AlgoG_Precision}
\end{figure}
%%%%%%%%%%%%%%%%%%%%%%%%%%

%%%%%%%%%%%%%%%%%%%%%%%%%%
\begin{figure}[tb]
	\begin{tabular}{@{}c@{}}
	\subfloat[a][]{
		\includegraphics[height=.45\textheight]{AlgoGene_RepartitionContacts_Carac2.eps}
		\label{fig:AlgoGene_Repartition_Carac2}}\\
	\subfloat[b][]{
		\includegraphics[height=.45\textheight]{AlgoGene_RepartitionContacts_Carac4.eps}
		\label{fig:AlgoGene_Repartition_Carac4}}\\
	\end{tabular}
	\caption{Repartition of the contact states of the solution obtained for $1000$ initialisations for the first set of parameters (top) and the second (bottom) on each frequency range - Blue: frictional state and yellow: frictional state}
	\label{fig:AlgoG_Repartition}
\end{figure}
%%%%%%%%%%%%%%%%%%%%%%%%%%





%%%%%%%%%%%%%%%%%%%%%%%%%%
\section{Conclusion}
%%%%%%%%%%%%%%%%%%%%%%%%%%
This numerical study showed the strong impact of several contacts between the caliper, the bracket, the pad and the piston on the propensity of automotive squeal.  It insights the difficulty to predict all instabilities with a unique configuration and highlights the need to consider various configurations in order to improve the prediction of squeal noise.
Even if the consideration of all contact conditions during the design process of brake systems is a necessity, this is not feasible nowadays because of the prohibitive number of calculations required. As a result, a complete strategy must be conducted to better undertake role of internal contacts on the squeal phenomena and to decide on the most influential contacts on the propensity of squeal. From an industrial practical point of view, this study demonstrates the feasibility of using Genetic Algorithm  to quickly find the most undesirable configuration (that can then be considered in a future design study) depending on the uncertainty and lack of control of contact conditions.




%%%%%%%%%%%%%%%%%%%%%%%%%%%%%%%%%%%%%%%%%%%%%%%%%%%%%%%%%%
\section*{Acknowledgments}

This work was achieved within PSA Peugeot Citro\"{e}n Stellab program - OpenLab Vibro-Acoustic-
Tribology@Lyon (VAT@Lyon).

J.-J. Sinou acknowledges the support of the Institut Universitaire de France.

\section*{References}

%\bibliographystyle{plain}
%\bibliographystyle{unsrt}
\bibliographystyle{acm}
%\bibliographystyle{abbrv}
\bibliography{biblioV2016}

\end{document}

%%% Local Variables:
%%% TeX-PDF-mode: t
%%% End:

